\begin{center}
\thispagestyle{empty}
\vspace{2cm}
\LARGE{\textbf{ABSTRACT}}\\[1.0cm]
\end{center}
\thispagestyle{empty}
\large{\paragraph{}
Land administration  and  title registration  system is the  system for  storing  land title  information  and  managing transactions involving land titles. Due to the sensitivity of land issues, land administration  and  title  registration system should be strong to avoid any document forgery, available all the time, and take a short time to complete tasks. Thus, this study aims at designing a model for such system based on blockchain technology. The proposed model is  designed using  UML  diagrams  and then tested for verification using statistical  usage models (Markov chains).  The  proposed  model  integrates  the  Integrated  Land  Management  Information  System  (ILMIS)  with factom  and  bitcoin  blockchains  which  enables  encryption  of  information  from  ILMIS  to  get  the  fingerprint information of  each  land title  and  store it  to  the  blockchains.  The  model further  encrypts  the  land information from  ILMIS  when  needed  and  then  compare  it  with  fingerprints  from  blockchains  for  verification.  Such implementation  of  the  proposed  model  will help  ILMIS  to  have the  capability  of  providing  tamper  proof  for stored  data,  providing  the  self-notarization  mechanism,  and  availability  of  evidence  for  the  land  title  from distributed  databases.  Furthermore,  the society  is  expected  to  benefit from this  study  as  the time  and cost  for registering land title will decrease and the possibilities of a piece of land having more than one owner will not be there. \\
}
\textbf{ \Large{ \\ Keywords:} } Blockchain, Bitcoin Blockchain,Etherium Blockchain,Security,Factom Blockchain, Land-Registry Management System.